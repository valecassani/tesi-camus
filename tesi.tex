%% LyX 2.1.4 created this file.  For more info, see http://www.lyx.org/.
%% Do not edit unless you really know what you are doing.
\documentclass[11pt,english,italian,openright]{book}
\usepackage[T1]{fontenc}
\usepackage[latin9]{inputenc}
\usepackage[a4paper]{geometry}
\geometry{verbose,tmargin=3cm,bmargin=3.5cm,lmargin=4cm,rmargin=3cm}
\setcounter{secnumdepth}{3}
\setcounter{tocdepth}{3}
\usepackage{color}
\usepackage{babel}
\usepackage{float}
\usepackage{booktabs}
\usepackage{url}
\usepackage{amsmath}
\usepackage{amssymb}
\usepackage{graphicx}
\usepackage{setspace}
\usepackage{esint}
\onehalfspacing
\usepackage[unicode=true,pdfusetitle,
 bookmarks=true,bookmarksnumbered=false,bookmarksopen=false,
 breaklinks=false,pdfborder={0 0 1},backref=false,colorlinks=false]
 {hyperref}

\makeatletter

%%%%%%%%%%%%%%%%%%%%%%%%%%%%%% LyX specific LaTeX commands.
\providecommand{\LyX}{\texorpdfstring%
  {L\kern-.1667em\lower.25em\hbox{Y}\kern-.125emX\@}
  {LyX}}
\DeclareRobustCommand*{\lyxarrow}{%
\@ifstar
{\leavevmode\,$\triangleleft$\,\allowbreak}
{\leavevmode\,$\triangleright$\,\allowbreak}}
%% Because html converters don't know tabularnewline
\providecommand{\tabularnewline}{\\}
\floatstyle{ruled}
\newfloat{algorithm}{tbp}{loa}[chapter]
\providecommand{\algorithmname}{Algoritmo}
\floatname{algorithm}{\protect\algorithmname}

%%%%%%%%%%%%%%%%%%%%%%%%%%%%%% User specified LaTeX commands.
% pacchetti aggiuntivi
\usepackage{tabularx}
\usepackage{setspace}
\usepackage{amsthm}
\usepackage{rotating}
\usepackage{caption}
\usepackage{epsfig}
\usepackage{indentfirst}
\usepackage{fancyhdr}
\usepackage{url}
\usepackage{cite}
\usepackage[normalem]{ulem}
\usepackage[table]{xcolor}
\usepackage{booktabs}
\usepackage{algpseudocode}

% questa � uno sporca soluzione per gestire il fatto che le versioni
% LTS di Ubuntu/Kubuntu hanno un pacchetto "geometry" vecchio
\include{geometry.sty}

% sistema il numero pagina nelle prime pagine dei capitoli
\fancypagestyle{plain}{
\fancyhead{}
\renewcommand{\headrulewidth}{0pt}
\renewcommand{\footrulewidth}{0pt}
\fancyfoot[OC]{\begin{flushright}\thepage\end{flushright}}
}

% header "carini" per la tesi
\fancyhead{}
\fancyhead[LE]{\slshape \nouppercase \leftmark}
\fancyhead[RO]{\slshape \nouppercase \rightmark}
\fancyfoot[EC]{\begin{flushleft}\thepage\end{flushleft}}
\fancyfoot[OC]{\begin{flushright}\thepage\end{flushright}}
\renewcommand{\headrulewidth}{0.4pt}
\setlength{\headheight}{14pt}

\@ifundefined{showcaptionsetup}{}{%
 \PassOptionsToPackage{caption=false}{subfig}}
\usepackage{subfig}
\makeatother

\usepackage{listings}
\addto\captionsenglish{\renewcommand{\algorithmname}{Algorithm}}
\addto\captionsenglish{\renewcommand{\lstlistingname}{Listing}}
\addto\captionsitalian{\renewcommand{\algorithmname}{Algoritmo}}
\addto\captionsitalian{\renewcommand{\lstlistingname}{Listato}}
\renewcommand{\lstlistingname}{Listato}

\begin{document}
\frontmatter
\pagestyle{empty}
\newgeometry{margin=3cm}\input{frontespizio/frontespizio.tex}\restoregeometry

\cleardoublepage{}

\begin{flushright}
\emph{Ad una persona veramente speciale\ldots{}
}\cleardoublepage{}
\par\end{flushright}


\chapter*{Ringraziamenti}

\thispagestyle{empty}\input{ringraziamenti.tex}\cleardoublepage{}


\chapter*{Sommario}

\thispagestyle{empty}\input{sommario.tex}\cleardoublepage{}


\chapter*{Abstract}

\thispagestyle{empty}\input{abstract.tex}\cleardoublepage{}\pagenumbering{roman}
\setcounter{page}{1}
\pagestyle{fancy}\tableofcontents{}\listoffigures
\listoftables
\listof{algorithm}{Elenco degli algoritmi}
\cleardoublepage{}\mainmatter
\renewcommand{\sectionmark}[1]{\markright{\thesection.\ #1}}
\renewcommand{\chaptermark}[1]{\markboth{\thechapter.\ #1}{}}


\chapter*{Introduzione\label{chap:introduzione}}

\addcontentsline{toc}{chapter}{Introduzione}
\markboth{Introduzione}{Introduzione}\input{introduzione.tex}


\chapter{Background\label{chap:background}}

\section{Context-Awareness}

abc

\section{Mobile Mashup}

abc

\section{Web Services}

abc

\section{Stato dell'arte}

abc


\chapter{Il Progetto CAMUS\label{chap:camus}}

Introduzione ...

\section{Integrazione contesto e mashup}

abc

\section{Caso di studio del turismo}

abc


\chapter{Metodologia\label{chap:metodologia}}

\section{Organizzazione del framework}

abc

\section{Creazione dell'ecosistema dei servizi}

abc

\section{Universal CDT}

abc

\section{Associazione dei servizi al CDT}

abc

\section{Mashup Design}

abc

\section{App Execution}

abc


\chapter{Implementazione\label{chap:implementazione}}

\section{Architettura e tecnologie utilizzate}

abc

\section{Flusso di una richiesta}

abc

\section{Endpoint GraphQL}

abc

\section{Interazioni con l'app mobile e le web app}

abc

\section{Componenti}

abc

\subsection{Context Manager}

abc

\subsection{Primary Service Selection}

abc

\subsection{Query Handler}

abc

\subsection{Bridge}

abc

\subsubsection{REST Bridge}

abc

\subsection{Response Aggregator}

abc

\subsection{Support Service Selection}

abc


\chapter{Valutazione delle performance\label{chap:performance}}

\section{Sistema di test e modello di simulazione}

abc

\section{Tempo di risposta ad una singola richiesta}

abc

\section{Tempo di risposta per richieste multiple}

abc

\subsection{Distribuzione fissa}

abc

\subsection{Distribuzione esponenziale}

abc


\chapter*{Conclusioni e sviluppi futuri\label{chap:conclusioni}}

\addcontentsline{toc}{chapter}{Conclusioni}
\markboth{Conclusioni}{Conclusioni}\input{conclusioni.tex}\bibliographystyle{plain}
\bibliography{bibliografia}
\addcontentsline{toc}{chapter}{Bibliografia}

\appendix

\chapter{Prima appendice\label{app:prima-appendice}}

\input{appendice-a/appendice-a.tex}


\chapter{Seconda appendice \label{app:seconda-appendice}}

\input{appendice-b/appendice-b.tex}


\chapter{Terza appendice\label{app:terza-appendice}}

\input{appendice-c/appendice-c.tex}\cleardoublepage{}
\end{document}
